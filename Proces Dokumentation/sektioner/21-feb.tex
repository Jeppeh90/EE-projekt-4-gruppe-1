% !TEX root = ../prj4procesdokumentation.tex
% SKAL STÅ I TOPPEN AF ALLE FILER FOR AT MASTER-filen KOMPILERES 
\chapter*{21 Feb 2017}

\section*{Agenda}

\textbf{Mødested: Shannon}  \\
\textbf{Inviteret til mødet: Alle} \\\\
\textbf{Dagsorden:}\\

\begin{itemize}
	\item Fokuspunkter \\
Hovedfokus skal være trin kobleren – vi har en transformer for vi kan teste de harmoniske, men fokus er et system som fungerer
Ved en transmissionslinje ser vi på spændingsfaldet – vi opbygger et system som består af modstande og spoler – impedanser som belastning – alt efter en bestemt belastning vælger vi at justerer spændingen – vi kan lave et simuleringssystem 
Undersøg hvordan løsningerne er implementeret i dag. 
Vi kan kontakte Energinet.dk eller Dansk Energi – vi kan få et skema og se hvordan forsyningsnettet ser ud: dispositionstransformer med 60/10. 
Vi udvikler et system – her kan vi se på de harmoniske. Vi måler på frekvenserne for at se på indholdet af de harmoniske. – altså observering. Dette kan vises på en skræm for at inddrage IKN. 
Spændingsniveau: bestemmer selv – der er mulighed for 230V
Signalgeneratoreren virker bed 20-30 V – det kunne

	\item Gennemgang af review foretaget på os
	\item Eventuelt andet
\end{itemize}

\section*{Referat}
\textbf{Deltager:} 

\textbf{Fraværende med afbud:} 

\textbf{Referent:} 




