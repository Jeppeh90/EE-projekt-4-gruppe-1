\documentclass[12pt,a4paper]{article}
\usepackage[utf8]{inputenc}


\begin{document}
\title{Forventningsafstemning}
\maketitle

\section{Gruppestruktur}
Gruppen deles ikke op på forhånd, men opstår naturligt i forhold til hvilke opgaver der er at tage fat i. Kollektiv: Alle skal vide hvad andre gruppemedlemmer arbejder med. \\
Fast ordstyrer/referent til dagsorden/mødeindkaldelser: Caroline

\section{Proces, møder, kommunikation, fildeling}
Sprint omdeling, hvor et enkelt sprint har længden 1-2 uger. \\
”Scrum”-møder afholdes efter hvert sprint \\
Stå-op-møder holdes 1-2 gange om ugen alt efter om der er noget at snakke om. Med udgangspunkt i en fast dag: tirsdag klokken 10:15-12:00. Efter opgavernes størrelser, så opsummeres der videre til hele gruppen. \\
Arbejdsburde: Ligeligt fordelt, være god til at bede om hjælp fra andre gruppemedlemmer. \\
At være på forkant i forhold til deadlines. Lav en god tidsplan og finde en reviewgruppe, således gruppen danner milestones. \\

\section{Værktøjer}
Trello: bruges til opgavefordeling med overskrifterne ”To do”, ”Igangværende”, ”Til review”, ”Done”
Git: til fildeling med et tilhørende repository  
Latex: til filskrivning og dokumentation. Gruppemedlemmer får intro til Latex
Facebook gruppe dannes til kommunikation af vigtig viden/links/dokumentationer mellem gruppemedlemmerne. 

\section{Interne regler}
Afbud: Senest dagen forinden aftale kan der meldes afbud, eller om morgen grundet sygdom/andet. \\
Individuelle personer søger selv for at bede om hjælp\\
Trello: ALLE skal opdatere Trello-board løbende/hver dag. Man skriver lige hvad man har lavet på opgaven/hvad man har problemer med. Notér hvor mange timer der er brugt på opgaverne. \\
Skriv løbende dokumentation!!!


\end{document}