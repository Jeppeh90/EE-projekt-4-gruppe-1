\documentclass[a4paper, 11pt,oneside,openany, danish]{memoir} % Starter dokumentet af klassen memoir


%%%%%%%%%%%%%%%%%%%%%%%
		       % PREAMBLE %			
%%%%%%%%%%%%%%%%%%%%%%%



% Papirstørrelse og margener
\usepackage[paper=a4paper, hmargin=1.1in, vmargin=1.1in]{geometry}

% Font encoding og sprog
\usepackage[T1]{fontenc}					% Output encoding
\usepackage[utf8]{inputenc}				% Input encoding
\usepackage[danish]{babel}				% Sprog (orddeling)
\renewcommand{\danishhyphenmins}{22} 	% bedre orddeling, minimum to tegn før og efter deling
\usepackage{lmodern}  					% gør underscores pænere
\usepackage{microtype} 					% laver micro ændringer i text for at udgå luft og orddeling


%% Forside text
%\usepackage{soul} % lege lege
%\sodef\an{}{0.2em}{.9em plus.6em}{1em plus.1em minus.1em}
%\newcommand\stext[1]{\an{\scshape#1}}

% Fyldetekst (Lorem ipsum)
\usepackage{blindtext}

% Tabeller
\usepackage{booktabs}
\usepackage{threeparttable}
\usepackage[tableposition=top]{caption}
\usepackage{tabularx}
\usepackage{multirow}					% For at lave pæne tabeller
\usepackage{hhline}						% For at lave endnu pænere tabller
\newcolumntype{C}{>{\let\newline\\\arraybackslash\hspace{0pt}}X}

%matematik
\usepackage{amsmath,amssymb,mathtools,bm}
\newcommand{\tsub}[1]{_{\textup{#1}}}
\def\doubleunderline#1{\underline{\underline{#1}}}
\usepackage[separate-uncertainty = true,multi-part-units=single]{siunitx}

% XColor: Farver
\usepackage[svgnames,dvipsnames,x11names]{xcolor}

% Figurer og floats
\usepackage[]{graphicx}
\graphicspath{{figurer/}}
\usepackage{placeins}
\usepackage{float}			% Muliggoer eksakt placering af floats, f.eks. \begin{figure}[H]

%%% Tegning af kasser
%\usepackage{calc,graphicx,color}
%\definecolor{mygreen}{rgb}{0,0.6,0}
%\definecolor{mygray}{rgb}{0.5,0.5,0.5}

% Biblatex til referencer
\usepackage[backend=bibtex]{biblatex}
\addbibresource{bibfil.bib}





% Hyper ref
\usepackage[ unicode=true, colorlinks=false, linktocpage=true, 
pdfborder={0 0 0}, pdfstartpage=1, pdfstartview=FitV, breaklinks=true,
pdfpagemode=UseNone, pageanchor=true, pdfpagemode=UseOutlines,
plainpages=false, bookmarksnumbered, bookmarksopen=true,
bookmarksopenlevel=1, hypertexnames=true, pdfhighlight=/O, urlcolor=Black,
linkcolor=Black, citecolor=Black]{hyperref}

% Clever ref
\usepackage{cleveref}



\settocdepth{subsection}
\setsecnumdepth{subsection}

% Sidetal
% Sidetal
\let\footruleskip\undefined
\usepackage{fancyhdr}
\usepackage{lastpage}
\pagestyle{fancy} 
\fancyhf{} 

\fancyhead[R]{\leftmark}
\fancyfoot[R]{\thepage \hspace{0.008in} af \pageref{LastPage}}

\fancypagestyle{}{
	\renewcommand{\headrulewidth}{0pt}
	\fancyhf{}
	\fancyfoot[R]{\thepage \hspace{0.008in} af \pageref{LastPage}}%
	
}


% Starten på dokumentet
\begin{document}


%%%%%%%%%%%%%%%%%%%%%%%
		       % FORSIDEN %			
%%%%%%%%%%%%%%%%%%%%%%%

% !TEX root = ../prj4projektdokumentation.tex
% SKAL STÅ I TOPPEN AF ALLE FILER FOR AT MASTER-filen KOMPILERES 
\thispagestyle{empty}
{\centering
	{\scshape\LARGE Aarhus Universitet \par}
	\vspace{1cm}
	{\scshape\Large 4. semesterprojekt gruppe 1\par}
	{\scshape\Large Projektdokumentation\par}
	\vspace{1.5cm}
	{\huge\bfseries Spændingsregulator\par}
	\vspace{2cm}
	{\Large
		201509249 - Caroline Møller Sørensen\\
		201611140 - Sophia Amailie Mortensen\\
		201505195 - Dennis Slot Larsen \\
		201505115 - Laurids Givskov Jørgensen\\
		201508333 - Søren Jensen\\
		13114 - Jeppe Hansen\\  }
	\vfill
	Vejleder\par
	Emir Pasic
	
	\vfill
	
	{\large \today\par}
	\par}




\frontmatter
%%%%%%%%%%%%%%%%%%%%%%%
             % RESUME & ABSTRACT %			
%%%%%%%%%%%%%%%%%%%%%%%



%%%%%%%%%%%%%%%%%%%%%%%
         % INDHOLDSFORTEGNELSE %			
%%%%%%%%%%%%%%%%%%%%%%%

\tableofcontents

%%%%%%%%%%%%%%%%%%%%%%%
                        % KAPITLER %			
%%%%%%%%%%%%%%%%%%%%%%%

\mainmatter
% !TEX root = ../prj4projektdokumentation.tex
% SKAL STÅ I TOPPEN AF ALLE FILER FOR AT MASTER-filen KOMPILERES 

\chapter{Introduktion}

\section{Problemformulering}
Når belastningerne i et distributionssystem ændres, vil spændingsniveauet variere. Det er vigtigt, at spændingsniveauet holdes stabilt. Hvordan sikres dette?
% !TEX root = ../prj4projektdokumentation.tex
% SKAL STÅ I TOPPEN AF ALLE FILER FOR AT MASTER-filen KOMPILERES 

\chapter{Projektbeskrivelse}

Formålet med dette projekt, er at opbygge et system, der simulerer det danske transmissionssystem. For energileverandører i Danmark er det et lovmæssigt krav, at spændingsforsyningen hos forbrugerne altid ligger på 230 volt $\pm$10 procent, og det ønskes at undersøge mulighederne for at opfylde dette.\\ 
I dette projekt vil fokus være på stykket fra distributionstransformer og ud til forbrugere/belastninger. Systemet skal bestå af en spændingsregulator (trinkobler), en distributionslinje og to eller flere varierende belastninger. Det ønskes at måle strøm, spænding, power factor og effektretning således, at spændingsregulatoren hele tiden kan holde spændingen på $\pm$ et givent niveau, selvom belastningen ændres. Normalt måles disse værdier ved distributionstransformeren, men i dette projekt ønskes det at måle hos hver enkelt belastning. På den måde fås en bedre overvågning af systemet og bedre mulighed for at observere hvilken betydning, f.eks. belastningens afstand til distributionstransformeren har for spændingsniveauet.\\ Systemet skal have to indstillinger – en til manuelt valg af spændingsniveau og en til automatisk valg af passende spændingsniveau.\\ 
Det ønskes desuden at kunne måle frekvensindholdet i systemet for at kunne observere et eventuelt indhold af harmoniske. De harmoniske i systemet er højfrekvente og vil afsætte varme i transformerne og dermed forkorte deres levetid. Det er derfor relevant at kende til indholdet af disse.\\ 
Det er et krav, at målte værdier i systemet vises på en skærm. 

\begin{figure}[htbp] % (alternativt [H])
	\centering
	\includegraphics[width=0.7\textwidth]{Figure/RigtBillede}
	\caption{Visuel fremvisning af system}
	\label{fig:Rigtbillede}
\end{figure}
% !TEX root = ../prj4projektdokumentation.tex
% SKAL STÅ I TOPPEN AF ALLE FILER FOR AT MASTER-filen KOMPILERES 


\chapter{Termliste}

\begin{table}[htbp]
	\centering
	\begin{tabular}{|l|l|}
		\hline
		\textbf{Term} 	& \textbf{Beskrivelse} \\\hline
		Trinskifter	& Transformer med variabelt omsætningsforhold \\\hline
		Centralt	& Ved trinskifter \\\hline
		Decentralt 	& Ved forbrugeren \\\hline
		
	\end{tabular}
	\caption{Termbeskrivelse}
	\label{tab:termbeskrivelsen}
	
\end{table}
\input{sektioner/kravspecifikation/kravspecifikation}
% !TEX root = ../../prj4projektdokumentation.tex
% SKAL STÅ I TOPPEN AF ALLE FILER FOR AT MASTER-filen KOMPILERES 

\section{Systembeskrivelse}

Systemet der er udviklet har til opgave at regulere spændingsniveauet på en distributionslinje, afhængigt af målingerne fra sensorer ved hver forbruger.
I dette projekt er det ikke muligt at realisere, derfor udvikles et produkt, der kan simulere scenariet. Det simuleres ved at skalere spændingsniveauet fra standardniveauet på 230V ned til 4V på distributionssiden. Dette gør det muligt at arbejde med en 8-trins transformer med specifikationen 24V/8-0V.\\
I prototypen er udviklet en impedans til simulering af en distributionslinje med en længde svarende til en typisk distributionslinje.\\
På distributionslinjen er så placeret et antal belastninger, der skal illustrere en husstand. Disse er designet således, at skalering passer med resten af systemet. Dette er rammen systemet skal arbejde indenfor. \\
En enkelt type måleenhed fremstilles, som placeres centralt   ved hver belastning. Disse måleenheder kan måle spænding, strøm, faseforskydning, frekvenser og harmoniske.  Systemets frekvens er 50Hz ligesom frekvensen på det danske elnet.\\
Data fra måleenhederne samles så i en styringsenhed, der har til opgave at regulere spændingsniveauet, så det altid ligger på 4V på distributionssiden ved at skifte trin på transformeren. Selve trinskiftet står trinskiftenheden for.\\
På styringsenheden er det muligt at observere de målte værdier på en touchskærm. Ved manuel styring er det på samme touchskærm, der kan skiftes trin på transformeren.\\
Systemet, der er fremstillet, er en simulering af et produkt, der kan løse problemet. Den overordnet struktur er dog tænkt sådan, at den kan skaleres op.\\




\subsection{Termliste}

\begin{table}[htbp]
\centering
\begin{tabular}{|l|l|}
\hline
\textbf{Term} 	& \textbf{Beskrivelse} \\\hline
Spændingsregaulator	& Består af en transformer og tilkoblet sensorer \\\hline
Trin 	& Trin af spændingsniveau \\\hline

\end{tabular}
\caption{Termbeskrivelse}
\label{tab:termbeskrivelsen}

\end{table}
% !TEX root = ../../prj4projektdokumentation.tex
% SKAL STÅ I TOPPEN AF ALLE FILER FOR AT MASTER-filen KOMPILERES 

\section{MoSCoW}

\begin{itemize}
\item{Systemet \textbf{skal} bestå af en trintransformer 24/0-8V}
\item{Systemet \textbf{skal} måle spænding, strøm, power factor og harmoniske centralt og decentralt}
\item{Systemet \textbf{skal} vise data på en skærm}
\item{Systemet \textbf{skal} simulere en distributionslinje og flere forbrugere}
\item{Systemet \textbf{skal} kunne reguleres manuelt}
\item{Systemet \textbf{burde} kunne reguleres automatisk}
\item{Systemet \textbf{kunne} have en log}
\item{Distributionslinjen \textbf{kunne} indeholde en decentral producent}
\item{Systemet \textbf{vil ikke} fjerne harmoniske} 
\end{itemize}


% !TEX root = ../prj4projektrapport.tex
% SKAL STÅ I TOPPEN AF ALLE FILER FOR AT MASTER-filen KOMPILERES 

\section{Funktionelle krav}

Funktionaliteten af systemet er beskrevet i tre use cases, som beskriver brugerens interaktion med systemet. Use casene vil her blive kort beskrevet, og et use case diagram er vist i Figur \ref{fig:UsecaseDiagram}.
 Den automatiske spændingsregulering vil ligeledes blive beskrevet og vist i et tilstandsdiagram på Figur \ref{fig:autoSTM}.
  Se dokumentation\footnote{Projektdokumentation, 3.3, Funktionelle krav} for uddybning af funktionelle krav. 

\subsubsection{Use case 1 - Start manuel styring}
Målet med denne use case er at ændre systemets tilstand fra automatisk til manuel mode. Dette gøres af brugeren, som vælger manuel styring på brugergrænsefladen. 

\subsubsection{Use case 2 - Stop manuel styring}
Målet med denne use case er at ændre systemets tilstand fra manuel til automatisk mode. Dette gøres af brugeren, som vælger automatisk styring på brugergrænsefladen. 

\subsubsection{Use case 3 - Skift trin}
Målet med denne use case er at skifte trin på trintransformeren. Forudsætningen for use casen er, at systemet er i manuel mode. Brugeren kan via brugergrænsefladen skifte trin op eller ned afhængigt af hvilket trin, der er aktivt. 

\subsubsection{Use case diagram}
\begin{figure}[H] % (alternativt [H])
	\centering
	\includegraphics[width=0.5\textwidth]{figure/UsecaseDiagram}
	\caption{Usecase Diagram}
	\label{fig:UsecaseDiagram}
\end{figure}

\subsubsection{Beskrivelse af automatisk mode}
Automatisk mode funktionen overvåger spændingen ved forbrugeren og skifter trin på trintransformeren, hvis spændingen falder/stiger med 10\%. 
\begin{figure}[H] % (alternativt [H])
	\centering
	\includegraphics[width=0.8\textwidth]{figure/STM}
	\caption{Beskrivelse af automatisk mode}
	\label{fig:autoSTM}
\end{figure}
% !TEX root = ../../prj4projektdokumentation.tex
% SKAL STÅ I TOPPEN AF ALLE FILER FOR AT MASTER-filen KOMPILERES 

\section{Ikke funktionelle krav}

\subsection{Trintransformer}
\begin{enumerate}
	\item Maks belastning af spændingsregulatoren er 20VA
	\item Nominel spænding på primær siden er 24VAC
	\item Nominel spænding på sekundær siden er 4 , 5 eller 6 afhængigt af trin VAC
	\item Skal minimum kunne levere 500mA	
\end{enumerate}

\subsection{Belastning}
\begin{enumerate}
	\item Modstandsværdi på 54$\Omega$ giver spændingsfald på 10\%, når spændingen fra regulatoren er 4V.
\end{enumerate}

\subsection{Måleenhed}
\label{subsec:ME}
Kravene til målingerne i Måleenheden, er opgivet i en procent, da det ikke er relevant for vores "proof of concept"\  at gå dybere ned i målenøjagtighed og præcision. 
\begin{enumerate}
	\item Måle spændingen ved trinskifteren og forbrugerne mellem 0 og 8 Vrms
	\item Måle spændingen med en præcision på $\pm$ 5\%
	\item Måle strømmen ved trinskifteren og forbrugerne mellem 0 og 500mA
	\item Måle strømmen med en præcision på $\pm$ 5\%
	\item Måle og beregne power factor med en præcision på $\pm$ 5$\%$
	\item Beregne THD med en præcision på $\pm$ 5$\%$
\end{enumerate}

\subsection{Kommunikation}
Kravene til kommunikationen gælder for hele kommunikationsvejen fra sensorer til brugergrænseflade. Disse krav er sat jf. "proof of concept"\, hvor fejlhåndtering ikke har prioritet. 
\begin{enumerate}
	\item Forsinkelsen på brugergrænsefladen ifht. ændringer i målte værdier må ikke overstige 2,5 sekund. 
	\item 95\% af alle sendte data skal være korrekte, og uden forstyrrelser. 
\end{enumerate}


\input{sektioner/Accepttestspecifikation/accepttestspecifikation}
% !TEX root = ../prj4projektrapport.tex
% SKAL STÅ I TOPPEN AF ALLE FILER FOR AT MASTER-filen KOMPILERES 

\section{Funktionelle krav}

Funktionaliteten af systemet er beskrevet i tre use cases, som beskriver brugerens interaktion med systemet. Use casene vil her blive kort beskrevet, og et use case diagram er vist i Figur \ref{fig:UsecaseDiagram}.
 Den automatiske spændingsregulering vil ligeledes blive beskrevet og vist i et tilstandsdiagram på Figur \ref{fig:autoSTM}.
  Se dokumentation\footnote{Projektdokumentation, 3.3, Funktionelle krav} for uddybning af funktionelle krav. 

\subsubsection{Use case 1 - Start manuel styring}
Målet med denne use case er at ændre systemets tilstand fra automatisk til manuel mode. Dette gøres af brugeren, som vælger manuel styring på brugergrænsefladen. 

\subsubsection{Use case 2 - Stop manuel styring}
Målet med denne use case er at ændre systemets tilstand fra manuel til automatisk mode. Dette gøres af brugeren, som vælger automatisk styring på brugergrænsefladen. 

\subsubsection{Use case 3 - Skift trin}
Målet med denne use case er at skifte trin på trintransformeren. Forudsætningen for use casen er, at systemet er i manuel mode. Brugeren kan via brugergrænsefladen skifte trin op eller ned afhængigt af hvilket trin, der er aktivt. 

\subsubsection{Use case diagram}
\begin{figure}[H] % (alternativt [H])
	\centering
	\includegraphics[width=0.5\textwidth]{figure/UsecaseDiagram}
	\caption{Usecase Diagram}
	\label{fig:UsecaseDiagram}
\end{figure}

\subsubsection{Beskrivelse af automatisk mode}
Automatisk mode funktionen overvåger spændingen ved forbrugeren og skifter trin på trintransformeren, hvis spændingen falder/stiger med 10\%. 
\begin{figure}[H] % (alternativt [H])
	\centering
	\includegraphics[width=0.8\textwidth]{figure/STM}
	\caption{Beskrivelse af automatisk mode}
	\label{fig:autoSTM}
\end{figure}
% !TEX root = ../../prj4projektdokumentation.tex
% SKAL STÅ I TOPPEN AF ALLE FILER FOR AT MASTER-filen KOMPILERES 

\section{Ikke funktionelle krav}

\subsection{Trintransformer}
\begin{enumerate}
	\item Maks belastning af spændingsregulatoren er 20VA
	\item Nominel spænding på primær siden er 24VAC
	\item Nominel spænding på sekundær siden er 4 , 5 eller 6 afhængigt af trin VAC
	\item Skal minimum kunne levere 500mA	
\end{enumerate}

\subsection{Belastning}
\begin{enumerate}
	\item Modstandsværdi på 54$\Omega$ giver spændingsfald på 10\%, når spændingen fra regulatoren er 4V.
\end{enumerate}

\subsection{Måleenhed}
\label{subsec:ME}
Kravene til målingerne i Måleenheden, er opgivet i en procent, da det ikke er relevant for vores "proof of concept"\  at gå dybere ned i målenøjagtighed og præcision. 
\begin{enumerate}
	\item Måle spændingen ved trinskifteren og forbrugerne mellem 0 og 8 Vrms
	\item Måle spændingen med en præcision på $\pm$ 5\%
	\item Måle strømmen ved trinskifteren og forbrugerne mellem 0 og 500mA
	\item Måle strømmen med en præcision på $\pm$ 5\%
	\item Måle og beregne power factor med en præcision på $\pm$ 5$\%$
	\item Beregne THD med en præcision på $\pm$ 5$\%$
\end{enumerate}

\subsection{Kommunikation}
Kravene til kommunikationen gælder for hele kommunikationsvejen fra sensorer til brugergrænseflade. Disse krav er sat jf. "proof of concept"\, hvor fejlhåndtering ikke har prioritet. 
\begin{enumerate}
	\item Forsinkelsen på brugergrænsefladen ifht. ændringer i målte værdier må ikke overstige 2,5 sekund. 
	\item 95\% af alle sendte data skal være korrekte, og uden forstyrrelser. 
\end{enumerate}


\input{sektioner/Arkitektur/Arkitektur}
% !TEX root = ../../prj4projektdokumentation.tex
% SKAL STÅ I TOPPEN AF ALLE FILER FOR AT MASTER-filen KOMPILERES 

\section{Blok definitionsdiagram}
Et BDD for spændingsregulator ses på figur \ref{fig:BDDSpaendingsregulator}. På diagrammet ses de overordnede blokke, spændingsregulator består af. En beskrivelse af hver blok kan læses under figur \ref{fig:BDDSpaendingsregulator}.

\begin{figure}[htbp] % (alternativt [H])
	\centering
	\includegraphics[width=0.9\textwidth]{Figure/BDDSpaendingsregulator}
	\caption{BDD Spændingsregulator}
	\label{fig:BDDSpaendingsregulator}
\end{figure}

\textbf{Måleenhed} står for at måle spænding, strøm og faseforskydningen herimellem. Ligeledes skal denne kunne måle indholdet af harmoniske frekvenser. Den skal bestå af hardware til måling af de nævnte parametre og en PSoC. På enheden skal behandlingen af rådataet også ligge, så dette kan formidles til styringsenheden.

\textbf{Styringsenhed} har til opgave at styre trinskifteren ud fra de data den får fra målenehderne. Den består af en PLC, der skal kommunikerer med brugergrænsefladen, så en bruger kan følge med i data fra Måleenhederne.

\textbf{Brugergrænsefladen} står for at formidle måledata til brugeren gennem en skærm, men det er også her at brugeren skal kunne interagere med systemet i manuel tilstand.

\textbf{Kommunikationsmodul} skaber en kommunikation fra Måleenhedens PSOC til Styringsenhedens PLC.

\textbf{Kontrolmodul} laves på en PLC, til at styrer trinene på trinskifteren.






\textbf{Trinskifter} er en enhed der kan skifte trin på en transformer ud fra et signal fra styringsenheden. Den skal altså bestå af et relæ for hvert trin, der kan kontrolleres af styringsenheden.
% !TEX root = ../../prj4projektrapport.tex
% SKAL STÅ I TOPPEN AF ALLE FILER FOR AT MASTER-filen KOMPILERES 


\section{Interne blokdiagramer}

Herunder ses de interne blokdiagrammer for spændingsregulator systemet. De interne blokdiagrammer er dannet for at give et indtryk af hvordan enhederne kommunikerer og sender signaler indbyrdes. Først ses det overordnede blokdiagram for hele systemet, hvor de overordnede forbindelser i spændingsregulatoren kan ses. Se figur \ref{fig:IBDSp}. Derefter ses et internt blokdiagram for selve styringsenhed, der yderligere består af tre mindre dele. 

\begin{figure}[htbp] % (alternativt [H])
	\centering
	\includegraphics[width=0.8\textwidth]{figure/IBDSpaendingsregulator.pdf}
	\caption{IBD for Spændingsregulator}
	\label{fig:IBDSp}
\end{figure}

Det interne blokdiagram for styringsenheden er udviklet for at beskrive hvordan modulerne i styringsenheden kommunikerer indbyrdes. Det ses samtidig hvilke inputs og outputs der er til den samlede styringsenhed. Se figur \ref{fig:IBDSp}. 


\begin{figure}[htbp] % (alternativt [H])
	\centering
	\includegraphics[width=0.8\textwidth]{figure/IBDStyringsenhed.pdf}
	\caption{IBD for Styringsenhed}
	\label{fig:IBDSt}
\end{figure} 
	 





%!TEX root = ../../prj4projektdokumentation.tex
% SKAL STÅ I TOPPEN AF ALLE FILER FOR AT MASTER-filen KOMPILERES 

\section{Allokeringsdiagram}
På figur \ref{fig:Allokering} ses allokeringsdiagram for spændningsregulatoren. Diagrammet er lavet for at danne overblik over softwaren, derfor er analog moduler undladt. Diagrammet viser hvilke platforme de logiske blokke skal laves på, og hvordan kommunikationen er mellem blokkene.

\begin{enumerate}
	\item Brugergrænsefladen allokeres på en HMI skærm.
	\item Kontrolmodulet allokeres på en PLC.
	\item Kommunikationsmodulet allokeres på en Arduino.
	\item Måleenhederne allokeres på PSoCs.
\end{enumerate}   
Kommunikationen mellem blokkene er allokeret på tre forskellige protokoller, se Kapitel \ref{ch:KomProtokol} for uddybning af disse.
\begin{enumerate}
	\item Mellem HMI og PLC anvendes Siemens standard PROFIBUS
	\item Mellem PLC og Arduino anvendes en TCP-protokol
	\item Mellem Arduino og PSOC anvendes en UART-protokol
\end{enumerate}

\begin{figure}[htbp] % (alternativt [H])
	\centering
	\includegraphics[width=0.9\textwidth]{Figure/Allokering}
	\caption{Allokeringsdiagram for spændingsregulator}
	\label{fig:Allokering}
\end{figure}

\newpage
% !TEX root = ../prj4projektdokumentation.tex
% SKAL STÅ I TOPPEN AF ALLE FILER FOR AT MASTER-filen KOMPILERES 

\chapter{Foranalyse}

\section{Valg af transformer}
Emir udleverede to transformere. Den ene var der ingen mærkeplade og ledninger var alle samme farve. Den anden har mærkeplade og forskellige farvet ledninger. De to transformere blev begge undersøgt i laboratoriet, med en 18VAC på primærside, derefter måltes spændingen på de forskellige trin. Transformeren uden mærkeplade hoppede små skridt fra ca 3 - 4.5V, hvor transformeren med mærkeplade havde skridt på 1 V fra 0 til 8V. Gruppen blev enige om at det var ligegyldigt om det var små eller store skridt transformeren hoppede, bare protypen blev lavet så den passede til transformeren. Det blev derfor besluttet at anvende transformeren med mærkeplade og forskellige farvet ledninger. 

\section{Valg af styrings enhed}
Det er besluttet at lave styringen på PLC da den virker ideel til at lave automationen til styring af trintransformeren, samtidig med at den kan få input fra sensorer og vise værdiger på en brugergrænseflade. Derudover kan kurset introduktion til automation også inddrages i projektet.

\section{Valg af sensorer}


\printbibliography

\end{document}


