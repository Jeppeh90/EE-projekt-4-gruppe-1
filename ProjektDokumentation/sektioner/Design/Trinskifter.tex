% !TEX root = ../../prj4projektdokumentation.tex


\section{Trinskifter}
For at kunne skifte trin og dermed spændingsniveau på tranformeren er der opbygget et kredsløb med tre kontaktrelæer - et for hvert muligt trin. Det er et krav, at relæerne skal kunne styres både automatisk og manuelt fra en PLC. Udgangssignalet fra PLC'en er 24VDC,så dette skal relæerne altså kunne holde til. På værkstedet var kun en type relæ der kan holde til 24VDC styresignal, og derfor faldt valget på disse Hengstler 468 relæer. På figuren nedenfor ses tegning af relækredsløbet samt de forskellige belastninger, der kan kobles ind via kontakter. I serie med hver belastningsmodstand sidder en 1 $\Omega$ modstand, som måleenheden måler over. 

%indsæt billede af relækredsløb.

Som vist på tegningen er hvert relæ forbundet til hver sin udgang på PLC, og det er herfra de modtager styresignalet. Hvert relæ er desuden forbundet til henholdsvis trin 4, 5 og 6. Gennem en Normally Open contact er hvert relæ videre forbundet ud til distributionslinjen. Ved skift af trin fra for eksempel 4 til 5 sendes højt signal til begge trin og hereftes slukkes signalet til trin 4. Der er altså et overlap ved skift af trin, og på den måde forsvinder forsyningen til belastningen ikke undervejs. 