% !TEX root = ../../prj4projektdokumentation.tex
\chapter{Design}
\section{Belastning}

Med denne distributionslinje og med trinskifteren, der står på 4V trinnet, kan det beregnes hvilken belastning, der vil medføre, at spændingen hos forbrugeren falder til under 10 \% \\ af 4V. Kredsløbet og beregningen ses nedenfor.

%indsæt billede af kredsløbet

Først bruges spændingsdelerformlen

\begin{align}
	V2=V1\cdot\frac{R_B}{Z_L+R_B}
\end{align}

\begin{align}
	0,9\cdot\vert V1 \vert = \vert V1\cdot\frac{R_B}{Z_L+R_B} \vert = V1\cdot\frac{R_B}{R_L+jX_L+R_B}
\end{align}

\begin{align}
0,9= \vert \frac{R_B}{R_L+jX_L+R_B} \vert
\end{align}