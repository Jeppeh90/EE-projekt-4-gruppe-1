% !TEX root = ../prj4projektdokumentation.tex
% SKAL STÅ I TOPPEN AF ALLE FILER FOR AT MASTER-filen KOMPILERES 

\chapter{Foranalyse}

\section{Valg af transformer}
Projektgruppens vejleder udleverede to transformere. Den ene er der ingen mærkeplade og ledningerne er alle samme farve. Den anden har mærkeplade og forskellige farvet ledninger. De to transformere blev begge undersøgt i laboratoriet, med en 18VAC på primærside, derefter måltes spændingen på de forskellige trin. Transformeren uden mærkeplade hoppede små skridt fra ca 3 - 4.5V, hvor transformeren med mærkeplade havde skridt på 1 V fra 0 til 8V. Gruppen blev enige om at det var ligegyldigt om det var små eller store skridt transformeren hoppede, bare protypen blev lavet så den passede til transformeren. Det blev derfor besluttet at anvende transformeren med mærkeplade og forskellige farvet ledninger. 

\section{Valg af styringsenhed}
Det er besluttet at lave Styringsenheden på en PLC, dette gøres for at realisere, hvordan det ville laves i virkeligheden. Alternativt kunne det laves på en PSOC eller Arduino, men da der på dette semester er undervisning i PLC styring virker det derfor oplagt.

\section{Overvejelser omkring måleenhederne}
Det var først tænkt at der kunne laves eller købes nogle sensorer som skulle tilsluttes PLC'en. PLC'en kan dog ikke måle hurtigt nok på AC forbindelser til at få et reelt billede af signalet. Så det kan undersøges for harmoniske svingninger. Ved at sample et signal på en PSOC, med en hurtigere samplefrekvens, kan der udregnes fourier. Fourier kan anvendes til at udregne størrelsen på et signal ved bestemte frekvenser, og derved kan systemet undersøges for harmoniske. Der findes en måde til at angive hvor stort et indhold af harmoniske, der er i et forsyningssystem, som kaldes Total Harmonic Distortion (THD). Dette er et tal i procent, som angiver hvor stort indholdet af harmoniske er. Da beregningen af dette er forholdsvis simpelt, vil denne metode blive brugt til at analyserer systemets indhold af harmoniske. 

\section{Simulering af distributionslinje}
For at opnå en virkelighedstro simulering af en distributionslinje blev der i starten af projektet taget kontakt til energiselskabet Eniig. Herfra modtog projektgruppen data på et udsnit af et distributionsnet. Forinden var der i samarbejde med vejleder fundet kabeldata på nkt cables hjemmeside. Afstandene mellem distributionstransformer og belastninger i Eniigs data er forholdsvis korte (100-500 meter). Med de fundne kabeldata vil der ikke være modstand- og spolevirkning, der vil påvirke spændingen i simuleringen. Velvidende at det ikke stemmer overens med virkeligheden, valgtes det derfor at simulere en distributionslinje på 60 kilometer. 

\section{Protokoller}



