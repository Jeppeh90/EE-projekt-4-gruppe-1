% !TEX root = ../../prj4projektdokumentation.tex


\section{Modultest}
\subsection{Spænding over belastning}
For at tjekke, at distributionslinjen og de forskellige belastninger giver det spændingsfald, der forventes laves en modultest af det samlede system med distributionslinje, belastninger og trinskifter. Der vælges forskellige belastninger og trin på transformeren, hvor for hver indstilling måles spændingsfaldet over belastningen. Resultaterne heraf ses i tabel \ref{tab:Modultesttrin}. 


\begin{table}[H]
	\centering
	\begin{tabular}{|l|l|l|l|}
		\hline
		\textbf{Trin} & \textbf{Belastning} & \textbf{Forventet resultat} & \textbf{Resultat} \\\hline
		
		\multirow{6}{*}{4 V AC} 
		& $50 \Omega\parallel 50 \Omega$ & 3,2 V AC over belastning & 2,9 V AC målt over belastning \\\hhline{~---} 	
		& $50 \Omega\parallel 16 \Omega$ & 2,6 V AC over belastning & 2,0 V AC målt over belastning \\\hhline{~---}
		& $54 \Omega$ & x V AC over belastning & x V AC målt over belastning \\\hhline{~---} 
		& $50 \Omega$ & 3,6 V AC over belastning & 3,5 V AC målt over belastning \\\hhline{~---} 	
		& $39 \Omega$ & 3,5 V AC over belastning & 3,4 V AC målt over belastning \\\hhline{~---} 
		& $16 \Omega$ & 2,8 V AC over belastning & 2,5 V AC målt over belastning \\\hline 
	
		
		\multirow{6}{*}{5 V AC} 
		& $50 \Omega\parallel 50 \Omega$ & 4,0 V AC over belastning & 3,7 V AC målt over belastning \\\hhline{~---} 	
		& $50 \Omega\parallel 16 \Omega$ & 3,22 V AC over belastning & 2,76 V AC målt over belastning \\\hhline{~---}
		& $54 \Omega$ & x V AC over belastning & x V AC målt over belastning \\\hhline{~---} 
		& $50 \Omega$ & 4,4 V AC over belastning & 4,6 V AC målt over belastning \\\hhline{~---} 	
		& $39 \Omega$ & 4,3 V AC over belastning & 4,2 V AC målt over belastning \\\hhline{~---} 
		& $16 \Omega$ & 3,5 V AC over belastning & 3,1 V AC målt over belastning \\\hline
		
	\end{tabular}
	\caption{Modultest for distributionslinje, belastninger og trinskifter}
	\label{tab:Modultesttrin}
\end{table}

Det kan dermed ud fra modultesten konkluderes, at spændingen over belastningerne falder som forventet, dog med en afvigelse på maksimalt 0,5V. Afvigelsen er størst, jo lavere belastning som tilsluttes. 


\subsection{Power factor}

For at teste at systemet virker som ønsket, laves en sammenligning af forventede power factor værdier og de værdier, der rent faktisk måles i systemet. Resultaterne herfra ses i tabel \ref{tab:Modultestpow}. Det ses at variationen i power factor er meget lille, og der ses derfor heller ingen forskel i de målte værdier. 
\begin{table}[H]
	\centering
	\begin{tabular}{|l|l|l|l|}
		\hline
		\textbf{Trin} & \textbf{Belastning} & \textbf{Simulering} & \textbf{Målt værdi} \\\hline
		
		\multirow{6}{*}{4 V AC} 
	
		& $54 \Omega$ & 0,997 &  0,999\\\hhline{~---} 
		& $50 \Omega$ &0,997  &  0,999\\\hhline{~---} 	
		& $39 \Omega$ & 0,996 &  0,999\\\hhline{~---} 
		& $16 \Omega$ & 0,982 &  0,999\\\hline 
		
		
		\multirow{6}{*}{5 V AC} 
	
		& $54 \Omega$ & 0,997 & 0,999 \\\hhline{~---} 
		& $50 \Omega$ & 0,997 & 0,999 \\\hhline{~---} 	
		& $39 \Omega$ & 0,996 & 0,999 \\\hhline{~---} 
		& $16 \Omega$ & 0,982 & 0,999 \\\hline 

		
	\end{tabular}
	\caption{Modultest for power factor}
	\label{tab:Modultestpow}
\end{table}