% !TEX root = ../../prj4projektdokumentation.tex
% SKAL STÅ I TOPPEN AF ALLE FILER FOR AT MASTER-filen KOMPILERES 

\section{Funktionelle krav}
I dette afsnit beskrives de funktionelle krav for systemet. Herunder usecase diagram og fullydressed usecases.

\subsection{Usecase Diagram}

\begin{figure}[htbp] % (alternativt [H])
	\centering
	\includegraphics[width=0.5\textwidth]{Figure/UsecaseDiagram}
	\caption{Usecase Diagram}
	\label{fig:UsecaseDiagram}
\end{figure}
Systemet indholder 3 usecases, der alle er initieret af brugeren. Den automatiske del af systemet er beskrevet i afsnit \ref{Automatisk mode}.

\subsection{Aktør Beskrivelse}
\textbf{Brugeren} er den primær aktør. Er en sikkerhedsgodkendt operatør der kan betjene trintransformeren manuelt.

\subsection{Usecase 1 - Start manuel styring}
\begin{table}[H]
	\centering
	
	\begin{threeparttable}
		\begin{tabularx}{\linewidth}{ l X }
			\toprule
			\bfseries{Navn:}				& UC1 - Start manuel styring  \\
			\midrule
			\bfseries{Mål:} 				& At sætte systemet i manuel mode \\
			\midrule
			\bfseries{Initiering:} 			& Initieres af brugeren. \\
			\midrule
			\bfseries{Aktører:} 			& Brugeren (Primær) \\
			\midrule
			\bfseries{Samtidige forekomster:} & 1 \\
			\midrule
			\bfseries{Forudsætninger:} 		& At systemet er funktionelt og i automatisk mode\\
			\midrule
			\bfseries{Resultat:} 			& Systemet er i manuel mode \\
			\midrule
			\bfseries{Hovedscenariet:} 	& \\
			
			
			1 	& Brugeren trykker "Manuel styring" på skærmen.\\
			2	& Systemet skifter til "Manuel mode" \\
			3 	& Systemet viser mulige spændingsniveauer på skærm 	\\
			
			\bottomrule
			
		\end{tabularx}
	\end{threeparttable}
	\caption{Fully dressed use case for UC1 - Start manuel styring}
	\label{table:UC1}
\end{table}

\subsection{Usecase 2 - Stop manuel styring}

\begin{table}[H]
	\centering
	
	\begin{threeparttable}
		\begin{tabularx}{\linewidth}{ l X }
			\toprule
			\bfseries{Navn:}				& UC2 - Stop manuel styring  \\
			\midrule
			\bfseries{Mål:} 				& At sætte systemet i automatisk mode \\
			\midrule
			\bfseries{Initiering:} 			& Initieres af brugeren. \\
			\midrule
			\bfseries{Aktører:} 			& Brugeren (Primær) \\
			\midrule
			\bfseries{Samtidige forekomster:} & 1 \\
			\midrule
			\bfseries{Forudsætninger:} 		& At systemet er funktionelt og i manuel mode\\
			\midrule
			\bfseries{Resultat:} 			& Systemet er i automatisk mode \\
			\midrule
			\bfseries{Hovedscenariet:} 	& \\
			
			
			1 	& Brugeren trykker "Autmatisk styring" på skærmen.\\
			2 	& Systemet skifter til automatisk mode.\\		
				
			
			\bottomrule
			
		\end{tabularx}
	\end{threeparttable}
	\caption{Fully dressed use case for UC2 - Stop manuel styring}
	\label{table:UC2}
\end{table}

\subsection{Usecase 3 - Skift trin}

\begin{table}[H]
	\centering
	
	\begin{threeparttable}
		\begin{tabularx}{\linewidth}{ l X }
			\toprule
			\bfseries{Navn:}				& UC3 - Skift trin  \\
			\midrule
			\bfseries{Mål:} 				& At skifte trin på transformeren \\
			\midrule
			\bfseries{Initiering:} 			& Initieres af brugeren. \\
			\midrule
			\bfseries{Aktører:} 			& Brugeren (Primær) \\
			\midrule
			\bfseries{Samtidige forekomster:} & 1 \\
			\midrule
			\bfseries{Forudsætninger:} 		& At systemet er funktionelt og i manuel mode\\
			\midrule
			\bfseries{Resultat:} 			& Transformerens trin er skiftet \\
			\midrule
			\bfseries{Hovedscenariet:} 	& \\
			
			
			1 	& Brugeren vælger trin på skærmen.\\
			2 	& Systemet skifter trin på transformeren.\\			
			
			\bottomrule
			
		\end{tabularx}
	\end{threeparttable}
	\caption{Fully dressed use case for UC3 - Skift trin}
	\label{table:UC3}
\end{table}


\subsection{Beskrivelse af automatisk mode}
\label{Afsnit: Automatisk mode}

\begin{figure}[htbp] % (alternativt [H])
	\centering
	\includegraphics[width=0.8\textwidth]{Figure/STM}
	\caption{Beskrivelse af automatisk mode}
	\label{fig:automode}
\end{figure}