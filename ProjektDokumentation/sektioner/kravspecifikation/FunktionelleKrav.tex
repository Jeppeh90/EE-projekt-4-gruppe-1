% !TEX root = ../../prj4projektdokumentation.tex
% SKAL STÅ I TOPPEN AF ALLE FILER FOR AT MASTER-filen KOMPILERES 

\section{Funktionelle krav}
I dette afsnit beskrives de funktionelle krav for systemet. De dele, hvor en bruger interager med systemet er beskrevet med usecase diagrammer. Den automatiske del er beskrevet og vist vha. et STM diagram.

\subsection{Beskrivelse af automatisk mode}
\label{Afsnit: Automatisk mode}

Når spændingsregulatoren er i automatisk mode, kontrolleres spænding ved forbrugerne. Hvis den spænding er for høj eller lav iht. de 4V skiftes der et trin op eller et trin ned.  
\begin{figure}[htbp] % (alternativt [H])
	\centering
	\includegraphics[width=0.8\textwidth]{Figure/STM}
	\caption{Beskrivelse af automatisk mode}
	\label{fig:automode}
\end{figure}

\subsection{Usecase Diagram}

\begin{figure}[H] % (alternativt [H])
	\centering
	\includegraphics[width=0.5\textwidth]{Figure/UsecaseDiagram}
	\caption{Usecase Diagram}
	\label{fig:UsecaseDiagram}
\end{figure}
Systemet indholder tre usecases, der alle er initieret af brugeren. Den automatiske del af systemet er beskrevet i afsnit \ref{Afsnit: Automatisk mode}.

\subsection{Aktør Beskrivelse}
\textbf{Brugeren} er den primær aktør. En sikkerhedsgodkendt operatør der kan betjene systemet.

\subsection{Usecase 1 - Start manuel styring}
\begin{table}[H]
	\centering
	
	\begin{threeparttable}
		\begin{tabularx}{\linewidth}{ l X }
			\toprule
			\bfseries{Navn:}				& UC1 - Start manuel styring  \\
			\midrule
			\bfseries{Mål:} 				& At sætte systemet i manuel mode \\
			\midrule
			\bfseries{Initiering:} 			& Initieres af brugeren. \\
			\midrule
			\bfseries{Aktører:} 			& Brugeren (Primær) \\
			\midrule
			\bfseries{Samtidige forekomster:} & 1 \\
			\midrule
			\bfseries{Forudsætninger:} 		& At systemet er funktionelt og i automatisk mode\\
			\midrule
			\bfseries{Resultat:} 			& Systemet er i manuel mode \\
			\midrule
			\bfseries{Hovedscenariet:} 	& \\
			
			
			1 	& Brugeren trykker Manuel styring på skærmen.\\
			2	& Systemet skifter til Manuel mode. \\
			3 	& Systemet aktivere manuel skærm. 	\\
			
			\bottomrule
			
		\end{tabularx}
	\end{threeparttable}
	\caption{Fully dressed use case for UC1 - Start manuel styring}
	\label{table:UC1}
\end{table}

\subsection{Usecase 2 - Stop manuel styring}

\begin{table}[H]
	\centering
	
	\begin{threeparttable}
		\begin{tabularx}{\linewidth}{ l X }
			\toprule
			\bfseries{Navn:}				& UC2 - Stop manuel styring  \\
			\midrule
			\bfseries{Mål:} 				& At sætte systemet i automatisk mode \\
			\midrule
			\bfseries{Initiering:} 			& Initieres af brugeren. \\
			\midrule
			\bfseries{Aktører:} 			& Brugeren (Primær) \\
			\midrule
			\bfseries{Samtidige forekomster:} & 1 \\
			\midrule
			\bfseries{Forudsætninger:} 		& At systemet er funktionelt og i manuel mode\\
			\midrule
			\bfseries{Resultat:} 			& Systemet er i automatisk mode \\
			\midrule
			\bfseries{Hovedscenariet:} 	& \\
			
			
			1 	& Brugeren trykker Automatisk styring på skærmen.\\
			2 	& Systemet skifter til Automatisk mode.\\
			3 	& Systemet aktivere automatisk skærm. 	\\		
				
			
			\bottomrule
			
		\end{tabularx}
	\end{threeparttable}
	\caption{Fully dressed use case for UC2 - Stop manuel styring}
	\label{table:UC2}
\end{table}

\subsection{Usecase 3a - Skift trin}

\begin{table}[H]
	\centering
	
	\begin{threeparttable}
		\begin{tabularx}{\linewidth}{ l X }
			\toprule
			\bfseries{Navn:}				& UC3a - Skift trin op  \\
			\midrule
			\bfseries{Mål:} 				& At skifte et trin op på transformeren \\
			\midrule
			\bfseries{Initiering:} 			& Initieres af brugeren. \\
			\midrule
			\bfseries{Aktører:} 			& Brugeren (Primær) \\
			\midrule
			\bfseries{Samtidige forekomster:} & 1 \\
			\midrule
			\bfseries{Forudsætninger:} 		& At systemet er funktionelt og i manuel mode\\
			\midrule
			\bfseries{Resultat:} 			& Transformerens trin er skiftet et trin op \\
			\midrule
			\bfseries{Hovedscenariet:} 	& \\
			
			
			1 	& Brugeren vælger Trin Op på skærmen.\\
			2 	& Systemet skifter et trin op på transformeren.\\
			3 	& Aktuelt trin vises på skærmen.\\
			4 	& Måleværdier opdateres på skærmen.\\		
			
			\bottomrule
			
		\end{tabularx}
	\end{threeparttable}
	\caption{Fully dressed use case for UC3 - Skift trin}
	\label{table:UC3}
\end{table}

\subsection{Usecase 3b - Skift trin}

\begin{table}[H]
	\centering
	
	\begin{threeparttable}
		\begin{tabularx}{\linewidth}{ l X }
			\toprule
			\bfseries{Navn:}				& UC3b - Skift trin ned  \\
			\midrule
			\bfseries{Mål:} 				& At skifte et trin ned på transformeren \\
			\midrule
			\bfseries{Initiering:} 			& Initieres af brugeren. \\
			\midrule
			\bfseries{Aktører:} 			& Brugeren (Primær) \\
			\midrule
			\bfseries{Samtidige forekomster:} & 1 \\
			\midrule
			\bfseries{Forudsætninger:} 		& At systemet er funktionelt og i manuel mode\\
			\midrule
			\bfseries{Resultat:} 			& Transformerens trin er skiftet et trin ned \\
			\midrule
			\bfseries{Hovedscenariet:} 	& \\
			
			
			1 	& Brugeren vælger Trin Ned på skærmen.\\
			2 	& Systemet skifter et trin ned på transformeren.\\
			3 	& Aktuelt trin vises på skærmen.\\
			4 	& Måleværdier opdateres på skærmen.\\			
			
			\bottomrule
			
		\end{tabularx}
	\end{threeparttable}
	\caption{Fully dressed use case for UC3 - Skift trin}
	\label{table:UC3}
\end{table}


