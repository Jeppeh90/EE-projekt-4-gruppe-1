\chapter{Accepttest}

% !TEX root = ../../prj4projektdokumentation.tex

\section{Funktionelle Krav}
%usecase 1
\begin{table}[htbp]
	\centering
	\begin{tabular}{|p{2cm}|p{3cm}|p{4cm}|p{4.5cm}|p{1cm}|}
		\hline
		\textbf{UC1} & \textbf{Handling} & \textbf{Forventet resultat} & \textbf{Resultat} &\textbf{OK} \\\hline
		Start manuel styring & Bruger vælger Manuel Mode & På skærmen vises Manuel Mode med mulighed for valg af trin & Skærmen viser Manuel Mode & \checkmark \\\hline
		
		
	\end{tabular}
	
	
\end{table}

%usecase2
\begin{table}[htbp]
	\centering
	\begin{tabular}{|p{2cm}|p{3cm}|p{4cm}|p{4.5cm}|p{1cm}|}
		\hline
		\textbf{UC2} & \textbf{Handling} & \textbf{Forventet resultat} & \textbf{Resultat} &\textbf{OK} \\\hline
		Stop manuel styring & Bruger vælger Automatisk Mode & Skærmen viser Automatisk Mode, hvor trinknapperne er deaktiverede & Der vises Automatisk Mode & \checkmark \\\hline
		
		
	\end{tabular}
\end{table}

% usecase3a
\begin{table}[htbp]
	\centering
	\begin{tabular}{|p{2cm}|p{3cm}|p{4cm}|p{4.5cm}|p{1cm}|}
		\hline
		\textbf{UC3a} & \textbf{Handling} & \textbf{Forventet resultat} & \textbf{Resultat} &\textbf{OK} \\\hline
		Skift trin op & Systemet er i Manuel Mode. Bruger vælger Trin Op på skærmen & Systemet skifter trin op på transformeren og skærmen opdateres med nye værdier. & Transformeren skifter trin efter 2s. Derefter opdateres spænding, strøm, pf og THD på skærmen inden 2s.  & \checkmark \\\hline
		
		
	\end{tabular}
\end{table}

%usecase3b
\begin{table}[H]
	\centering
	\begin{tabular}{|p{2cm}|p{3cm}|p{4cm}|p{4.5cm}|p{1cm}|}
		\hline
		\textbf{UC3b} & \textbf{Handling} & \textbf{Forventet resultat} & \textbf{Resultat} &\textbf{OK} \\\hline
		Skift trin ned & Systemet er i Manuel Mode. Bruger vælger Trin Ned på skærmen & Systemet skifter trin ned på transformeren og skærmen opdateres med nye værdier. & Transformeren skifter trin efter 2s. Derefter opdateres spænding, strøm, pf og THD på skærmen inden 2s.  & \checkmark \\\hline
		
		
	\end{tabular}
	
	
\end{table}

\section{Ikke funktionelle krav}
% for spændingsregulator
\begin{table}[H]
	\centering
	\begin{tabular}{|p{4cm}|p{3cm}|p{3cm}|p{3cm}|p{1cm}|}
		\hline
		\textbf{Trintransformer} & \textbf{Handling} & \textbf{Forventet resultat} & \textbf{Resultat} &\textbf{OK} \\\hline
		Nominel spænding på primærsiden er 24VAC & Spændingen på primærsiden måles. & Den målte værdi er 24VAC. & Der måles 23.88 VAC & \checkmark \\\hline
		Nominel spænding på sekundær side er 4, 5 eller 6 VAC afhængig af trin & Spændingen på sekundærsiden måles for hhv. trin 4, 5 og 6. & Den målte værdi er 4, 5 eller 6 VAC alt efter valgt trin. & Der måles hhv. 4, 5 og 6 VAC & \checkmark \\\hline
		Skal minimum kunne levere 500mA	. & Strømmen på sekundærsiden måles. & Transformeren kan levere over 500mA & Ikke opfyldt  & \\\hline
	\end{tabular}
	
\end{table}

% for belastning
\begin{table}[H]
	\centering
	\begin{tabular}{|p{4cm}|p{3cm}|p{3cm}|p{3cm}|p{1cm}|}
		\hline
		\textbf{Belastning} & \textbf{Handling} & \textbf{Forventet resultat} & \textbf{Resultat} &\textbf{OK} \\\hline
		Modstandsværdi på 54$\Omega$ giver spændingsfald på 10\%, når spændingen fra regulatoren er 4V. & Den givne modstand indsættes som belastning, og spændingen herover måles. & Spændingen over belastningen måles til 3,6V. & 3,4V & \checkmark \\\hline	
	\end{tabular}
	
	
\end{table}

% for måleenhed
%\begin{table}[htbp]
%	\centering
\begin{longtable}{|p{4cm}|p{3cm}|p{3cm}|p{3cm}|p{1cm}|}
	\hline
	\textbf{Måleenhed} & \textbf{Handling} & \textbf{Forventet resultat} & \textbf{Resultat} &\textbf{OK} \\\hline
	Måle spændingen ved trinskifteren og forbrugerne mellem 0 og 8 Vrms & Måleenheden testes med spændinger fra 0 til 8Vrms, i intervaller af 500mV. & Korrekt spændingsmåling i hele intervallet. & Korrekte spændingsmålinger i hele intervallet & \checkmark\\\hline
	Måle spændingen med en præcision på $\pm$ 5\%& Måleenheden påtrykkes en spænding på 3,5Vrms. Der laves herefter ti målinger& Gennemsnits afvigelsen forventes at være under $\pm$5\%.& Afvigelsen er fundet til 1,6\%. &\checkmark\\\hline
	Måle strømmen ved trinskifteren og forbrugerne mellem 0 og 500mA& Måleenheden testes med strømme fra 0 til 500mA i intervaller af 50mA&Korrekt strømmåling i hele intervallet.&Korrekt strømmåling i hele intervallet&\checkmark\\\hline
	Måle strømmen med en præcision på $\pm$ 5\%&Måleenheden påtrykkes en spænding på 300mVrms (Svarende til 300mA), der laves herefter ti målinger&Gennemsnits afvigelsen forventes at være under $\pm$5\%&Afvigelsen er fundet til 0,8\%.&\checkmark\\\hline
	Måle og beregne power factor med en præcision på $\pm$ 5$\%$&Måleenheden måler power factor over en belastning på distributionslinjen, der sammenlignes med beregnet power factor&Afvigelsen forventes at være under $\pm$ 5\% &Afvigelsen er fundet til 2\% &\checkmark \\\hline
	Beregne THD med en præcision på $\pm$ 5$\%$&Måleenheden påtrykkes en firkantsignal med 1V amplituder og 1V offset. Der sammenlignes med beregnet THD for firkantsignal& Afvigelsen forventes at være under $\pm$ 5\%&Afvigelsen er fundet til 0,5\%&\checkmark\\\hline
	
\end{longtable}


%\end{table}