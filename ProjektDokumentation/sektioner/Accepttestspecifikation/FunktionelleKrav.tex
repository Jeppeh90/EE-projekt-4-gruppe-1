% !TEX root = ../prj4projektdokumentation.tex

\section{Funktionelle Krav}
%usecase 1
\begin{table}[htbp]
	\centering
	\begin{tabular}{|p{2cm}|p{3.5cm}|p{4cm}|p{2cm}|p{2.5cm}|}
		\hline
		\textbf{UC1} & \textbf{Handling} & \textbf{Forventet resultat} & \textbf{Resultat} &\textbf{Kommentar} \\\hline
		Start manuel styring & Bruger vælger manuel mode & Systemet er i manuel mode & xxx & xxx \\\hline
		
		
	\end{tabular}
	\caption{Test use case 1}
	\label{tab:TestUC1}
	
\end{table}

%usecase2
\begin{table}[htbp]
	\centering
	\begin{tabular}{|p{2cm}|p{3.5cm}|p{4cm}|p{2cm}|p{2.5cm}|}
		\hline
		\textbf{UC2} & \textbf{Handling} & \textbf{Forventet resultat} & \textbf{Resultat} &\textbf{Kommentar} \\\hline
		Stop manuel styring & Bruger vælger automatisk mode & Systemet er i automatisk mode & xxx & xxx \\\hline
		
		
	\end{tabular}
	\caption{Test use case 1}
	\label{tab:TestUC1}
	
\end{table}
% usecase3
\begin{table}[htbp]
	\centering
	\begin{tabular}{|p{2cm}|p{3.5cm}|p{4cm}|p{2cm}|p{2.5cm}|}
		\hline
		\textbf{UC3} & \textbf{Handling} & \textbf{Forventet resultat} & \textbf{Resultat} &\textbf{Kommentar} \\\hline
		Skift trin & Bruger vælger trin på skærm & Systemet skifter trin på transformer & xxx & xxx \\\hline
		
		
	\end{tabular}
	\caption{Test use case 1}
	\label{tab:TestUC1}
	
\end{table}