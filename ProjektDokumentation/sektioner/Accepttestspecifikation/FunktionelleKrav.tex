% !TEX root = ../../prj4projektdokumentation.tex

\section{Funktionelle Krav}
%usecase 1
\begin{table}[htbp]
	\centering
	\begin{tabular}{|p{2cm}|p{3cm}|p{4cm}|p{4.5cm}|p{1cm}|}
		\hline
		\textbf{UC1} & \textbf{Handling} & \textbf{Forventet resultat} & \textbf{Resultat} &\textbf{OK} \\\hline
		Start manuel styring & Bruger vælger Manuel Mode & På skærmen vises Manuel Mode med mulighed for valg af trin &  &  \\\hline
		
		
	\end{tabular}

	
\end{table}

%usecase2
\begin{table}[htbp]
	\centering
	\begin{tabular}{|p{2cm}|p{3cm}|p{4cm}|p{4.5cm}|p{1cm}|}
		\hline
		\textbf{UC2} & \textbf{Handling} & \textbf{Forventet resultat} & \textbf{Resultat} &\textbf{OK} \\\hline
		Stop manuel styring & Bruger vælger Automatisk Mode & Skærmen viser Automatisk Mode, hvor trinknapperne er deaktiverede & &  \\\hline
		
		
	\end{tabular}
\end{table}

% usecase3a
\begin{table}[htbp]
	\centering
	\begin{tabular}{|p{2cm}|p{3cm}|p{4cm}|p{4.5cm}|p{1cm}|}
		\hline
		\textbf{UC3a} & \textbf{Handling} & \textbf{Forventet resultat} & \textbf{Resultat} &\textbf{OK} \\\hline
		Skift trin op & Systemet er i Manuel Mode. Bruger vælger Trin Op på skærmen & Systemet skifter trin op på transformeren og skærmen opdateres med nye værdier. &   & \\\hline
		
		
	\end{tabular}
\end{table}

%usecase3b
\begin{table}[H]
	\centering
	\begin{tabular}{|p{2cm}|p{3cm}|p{4cm}|p{4.5cm}|p{1cm}|}
		\hline
		\textbf{UC3b} & \textbf{Handling} & \textbf{Forventet resultat} & \textbf{Resultat} &\textbf{OK} \\\hline
		Skift trin ned & Systemet er i Manuel Mode. Bruger vælger Trin Ned på skærmen & Systemet skifter trin ned på transformeren og skærmen opdateres med nye værdier. &  & \\\hline
		
		
	\end{tabular}
	
	
\end{table}