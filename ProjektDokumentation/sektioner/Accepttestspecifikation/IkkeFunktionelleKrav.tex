% !TEX root = ../../prj4projektdokumentation.tex
\section{Ikke funktionelle krav}
% for spændingsregulator
\begin{table}[htbp]
	\centering
	\begin{tabular}{|p{4cm}|p{3cm}|p{4cm}|p{3.5cm}|p{1cm}|}
		\hline
		\textbf{Spændingsregulator} & \textbf{Handling} & \textbf{Forventet resultat} & \textbf{Resultat} &\textbf{OK} \\\hline
		Nominel spænding på primærsiden er 24VAC & Spændingen på primærsiden måles. & Den målte værdi er 24VAC. & Der måles 23.88 VAC & \checkmark \\\hline
		Nominel spænding på sekundær side er 4,5 eller 6 VAC afhængig af trin & Spændingen på sekundærsiden måles for hhv. trin 4, 5 og 6. & Den målte værdi er 4, 5 eller 6 VAC alt efter valgt trin. & Der måles hhv. 4, 5 og 6 VAC & \checkmark \\\hline
		Maks strøm på sekundær side er 1 A. & Strømmen på sekundærsiden måles. & Den målte værdi overstiger ikke 1A. & xx  & xx \\\hline
	\end{tabular}
	
\end{table}

% for belastning
\begin{table}[htbp]
	\centering
	\begin{tabular}{|p{4cm}|p{3cm}|p{4cm}|p{2cm}|p{2.5cm}|}
		\hline
		\textbf{Belastning} & \textbf{Handling} & \textbf{Forventet resultat} & \textbf{Resultat} &\textbf{Kommentar} \\\hline
		Modstandsværdi på 43 $\Omega$ giver spændingsfald på 10\%, når spændingen fra regulatoren er 4V. & Den givne modstand indsættes som belastning, og spændingen herover måles. & Spændingen over belastningen måles til 3,6V. & xxx & xxx \\\hline	
	\end{tabular}

	
\end{table}
% for brugergrænseflade
\begin{table}[htbp]
	\centering
	\begin{tabular}{|p{4cm}|p{3cm}|p{4cm}|p{2cm}|p{2.5cm}|}
		\hline
		\textbf{Brugergrænseflade} & \textbf{Handling} & \textbf{Forventet resultat} & \textbf{Resultat} &\textbf{Kommentar} \\\hline
		Punkt ?? & xxx & xxx & xxx & xxx \\\hline
		Punkt?? & xx & xx & xx & xx \\\hline
		
		
	\end{tabular}

	
\end{table}