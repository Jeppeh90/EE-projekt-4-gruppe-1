% !TEX root = ../../prj4projektdokumentation.tex

\subsection{Fourier algoritme}
Når Strøm og spænding er samplet. Laves der Fourier på dem hver især vha. funktionen \ref{FFTproto}

\lstset{caption={FFT prototype},label={FFTproto}}
\begin{lstlisting}
void FFT(long m,double *x,double *y, double *u);
\end{lstlisting}

Denne funktion dækker over en algoritme der kan lave FFT\footnote{Fast Fourier Transformation} på et samplet signal. Algoritmen er fundet på et Cypress forum\cite{FFTalgo}. Funktionen kan lave FFT på et array med 64 tal, hvor den laver arrayet om til de reale tal og fylder et andet array med de imaginær tal. Derudover fylder den et tredje array med absolute amplitude værdiger. Det første argument m er eksponent til 2, så det giver størrelsen af arrayet. Ved 64 bliver eksponent 6.

Algoritmen starter med at lave noget der kaldes "bit-reversal", Dette betyder at man bytter pladserne i arrayet på følgende måde:

\begin{table}[H] 
	\centering 
	\begin{tabular}{|l|l|l|l|l|l|} % Afstem antal tegn og kolonner! (l for venstre, c for center, r for højre, | for lodret streg) 
		\hline 	% Vandret streg
		Array plads		& array plads binær & $\rightarrow$ 	& array plads binær    	& Array plads 	 \\ \hline 	% Linjeskift og vandret streg
		 	&   			& $\rightarrow$ & 				& 	 \\ \hline 
		 	&    			& $\rightarrow$ &					& 		 \\ \hline 
	\end{tabular} 
	\caption{X} 
	\label{tab:X} 
\end{table}
 


Ved samplingstiden der svare til at der kan måles over en periode på 50Hz og der samples 64 gange bliver samplefrekvensen:

\begin{align}
f_{sample} = 64 * 50Hz = 3200Hz
\end{align}   

Hvilket betyder at hver bin i array'ne giver:

\begin{align}
f_{Bin} = \dfrac{3200Hz}{64} = 50
\end{align}

Dette betyder at på plads nummer 2 i arrayet med absolute værdiger, ligger størrelsen på de signaler der har en frekvens på 50Hz og på plads nummer 3, signaler med 100Hz osv.

\subsection{Beregning af rms og power faktor}

\subsubsection{Beregning af rms ved 50Hz}
Rms kan per definition beregnes for strøm og spænding på ved følgende: 
\begin{align}
 X_{rms}= \dfrac{X_{amplitude}}{\sqrt{2}}
\end{align}
Derfor kan rms værdigen ved 50Hz beregnes med følgende funktion:
\lstset{caption={Funktion til beregning af 50Hz rms},label={rmsfunk}}
\begin{lstlisting}
double calculate_50Hz_RMS(double *u)
{
return u[1]/sqrt(2);  
}
\end{lstlisting}

Hvor "u" er et array med absolute værdiger for spænding eller strøm med frekvensbin på 50.

\subsubsection{Beregning af Power faktor ved 50Hz}

Power faktor kan er definition udregnes som cosinus til vinklen mellem strøm og spænding. Ved at lave Fourier på signalet, kan det komplekse tal anvendes til at lave en vektor. Power faktor for signalet på 50Hz bliver derfor vinklen mellem spænding og strøm på andet plads i arrayet.

Vinklen på et kompleks tal er per definition:

\begin{align}
\delta = atan(\dfrac{Imaginaire}{Real})
\end{align}

Power faktor kan derfor regnes som:
\begin{align}
PF = cos(\delta - \beta)
\end{align}

Måleenheden kan ikke se forskel på om det er et lag eller lead system. Det forventes næste altid at systemet har mere induktiv end kapacitiv belastning. Derudover betyder det heller ikke noget for effektiviteten om det er lag eller lead. Derfor er det blevet ned prioriteret på måleeneheden.

\lstset{caption={Funktion til beregning af 50Hz PF},label={PFfunk}}
\begin{lstlisting}
double calculate_50Hz_PF()
{
double angle_Volt = atan(Im_volt[1]/Re_volt[1]);
double angle_Ampere = atan(Im_Ampere[1]/Re_Ampere[1]);    
return cos(angle_Volt-angle_Ampere);
}
\end{lstlisting}
 
    

  
