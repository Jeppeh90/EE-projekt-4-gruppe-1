% !TEX root = ../../prj4projektdokumentation.tex

\section{Modultest}

Efter udvikling af Måleenheden er der lavet test på modulet, for at sikre den lever op til  kravene.

\subsection*{Test af spændingsmåling}
\begin{center}
	\begin{tabular}{ | m{0.2\textwidth} | m{0.8\textwidth}|} 
		\hline
		\textbf{Test}					&Spændingsmåling \\ \hline
		\textbf{Testbeskrivelse}		&Der måles ved 3 forskellige spændingsniveauer for at sikre at systemet kan måler i hele måleområdet  \\ \hline
		\textbf{Input}					&1V, 4V og 8V\\ \hline
		\textbf{Forventet output}		&1V, 4V og 8V\\ \hline
		\textbf{Resultat}				&\\ \hline
	\end{tabular}
\end{center}


\subsection*{Test af strømmåling}
\begin{center}
	\begin{tabular}{ | m{0.2\textwidth} | m{0.8\textwidth}|} 
		\hline
		\textbf{Test}					&Strømmåling \\ \hline
		\textbf{Testbeskrivelse}		&Der måles ved 3 forskellige strømniveauer for at sikre at systemet kan måler i hele måleområdet  \\ \hline
		\textbf{Input}					&0,1V, 0,3V og 0,5V\\ \hline
		\textbf{Forventet output}		&0,1A, 0,3A og 0,5A \\ \hline
		\textbf{Resultat}				&\\ \hline
	\end{tabular}
\end{center}


\subsection*{Test af power faktor}
\begin{center}
	\begin{tabular}{ | m{0.2\textwidth} | m{0.8\textwidth}|} 
		\hline
		\textbf{Test}					&Power faktor måling \\ \hline
		\textbf{Testbeskrivelse}		&Der måles om Power faktor stemmer overens med den enlige Power faktor. Dette gøres vinklen mellem strøm og spænding på oscilloskop, og derefter regnes power faktor.   \\ \hline
		\textbf{Input}					&2 sinus signaler med faseforskydning på \\ \hline
		\textbf{Forventet output}		&Power faktor på \\ \hline
		\textbf{Resultat}				&Power faktor på \\ \hline
	\end{tabular}
\end{center}   


\subsection*{Test af THD måling}
Test af THD kan ses i kapitel \ref{sek:THD}. 