% !TEX root = ../prj4projektrapport.tex
% SKAL STÅ I TOPPEN AF ALLE FILER FOR AT MASTER-filen KOMPILERES 


I dette afsnit vil overvejelser omkring hvad Spændingsregulatoren kan bidrage med til det eksisterende distributionsnet blive beskrevet. Ligeledes vil overvejelser omkring styrker, svagheder og relevans af Spændingsregulatoren blive diskuteret. Her er det konceptet for Spændingsregulatoren, der diskuteres og ikke selve prototypen, der er udviklet. 

Dette projekt er udarbejdet med udgangspunkt i problemformuleringen, men for at få en bedre forståelse for, hvordan denne problemstilling løses i dag, tog projektgruppen kontakt til energiselskabet Eniig. Efter besøg på en af deres transformerstationer viste det sig, at den regulering, der i projeket, laves på distributionssiden, ikke foretages her i virkeligheden. Her laver man i stedet trinskift ved 60/10 kV transformere og opretholder på denne måde en stabil forsyning hos forbrugerne. Som elnettet ser ud nu, kunne det derfor godt virke overflødigt at installere Måleenheder hos hver enkelt forbruger, da deres krav allerede er opfyldt. Hvis man kigger længere ud i fremtiden kunne man dog godt forestille sig et andet scenarie. Hvis eksempelvis flere og flere danskere begynder at køre elbiler, der skal oplades med jævne mellemrum, ville dette medføre endnu større og mere varierende belastning end i dag. Her ville Spændingsregulatoren give mulighed for at overvåge tilstanden hos den enkelte forbruger og regulere, således at det påkrævede spændingsniveau opretholdes. 
Med udgangspunkt i måling af værdier hos forbrugerne kunne man også forestille sig, at Spændingsregulatoren kunne have relevans i forhold til implementering af smart grid. På denne måde kan Spændingsregulatoren bidrage til optimal udnyttelse af tilgængelige ressourcer og optimering af elnettet. 

Foruden mere variation i belastningen er det også klart, at der i fremtiden vil være endnu flere decentrale producenter, eksempelvis i form af solcelleanlæg og vindmøller. Her kunne man forestille sig, at harmoniske ville blive et problem, der kunne medføre unødigt slid på transformere. Efter snak med Eniig viser det sig dog, at dette problem ikke er så stort i distributionssystemer. De decentrale producenter medfører til gengæld en udfordring i forhold til overspænding. Disse overspændinger kan medføre overbelastning og varme i kabler og udstyr. Denne problematik er dog ikke behandlet i prototypen, men ville være relevant at undersøge.

I forhold til selve implementeringen af Spændingsregulatoren ligger også nogle udfordringer. Først og fremmest vil det kræve en del udvikling af prototypen, inden den kan integreres på det eksisterende elnet. Her tænkes både på, overholdelse af gældende standarder, sikkerhed for forbrugere og økonomi. Udnyttelse af Spændingsregulatoren vil kræve, at der installeres en Måleenhed hos hver enkelt forbruger, men dette kan medføre høje omkostninger i forbindelse med produktionen. Disse omkostninger skal holdes op mod, hvad man mener at kunne vinde på implementeringen af konceptet. 


