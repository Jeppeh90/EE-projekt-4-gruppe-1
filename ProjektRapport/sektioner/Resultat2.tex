% !TEX root = ../prj4projektrapport.tex
% SKAL STÅ I TOPPEN AF ALLE FILER FOR AT MASTER-filen KOMPILERES 

Den automatisk udviklede regulering er lavet ud fra en meget simpel model, hvor der skiftes lige så snart spændingen når et bestemt niveau. Det kunne være lavet bedre med hystereser, som gjorde at spændingen skulle være ved et bestemt niveau i noget tid, inden den skiftede. Det ville kunne give et mere stabilt system, da Spændingsregulatoren ville stå og skifte op og ned ved små ændringer omkring skiftene. Det er dog ikke noget problem i projektets simulerede belastninger da de springer i lidt større trin og ikke variere hele tiden.

Forbrugerne er dimensioneret efter at kunne lave en maks. strøm i systemet på 500mA, dette har dog vist sig ikke at kunne lade sig gøre, da Spændingsregulatoren ikke kan opretholde 4V ved forbrugerne, hvis de alle bliver slået til. Det er dog ikke prioriteret at gøre noget ved, da det har synes at være irrelevant. Det ville kunne have været løst ved at inddrage flere trin på transformeren.

Der er blevet lavet en prototype, der med stor tilfredshed har kunne simulere en løsning på problemformuleringen. I forhold til accepttesten\footnote{Projektdokumentation, 12, Accepttest}, har Spændingsregulatoren bevist at der kan skiftes trin på transformeren manuelt eller automatisk iht. hvor stor belastning der er i systemet. Derudover har den vist sig at kunne måle strøm og spænding i det simulerede systemet med høj præcision.