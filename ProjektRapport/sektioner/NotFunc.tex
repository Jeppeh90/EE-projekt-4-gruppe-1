% !TEX root = ../prj4projektrapport.tex
% SKAL STÅ I TOPPEN AF ALLE FILER FOR AT MASTER-filen KOMPILERES 

\section{Ikke Funktionelle Krav}
I dette afsnit beskrives de ikke funktionelle krav. Der findes ikke funktionelle krav til trintransformeren, målingerne og kommunikation. Disse beskrives hver især i følgende afsnit.

\subsubsection{Trintransformer}
Der er valgt en trintransformer, som simulering af belastning skal laves ud fra. Denne valgt trintransformer har følgende specifikationer. Det er dog besluttet ikke at lave større strømforbrug i systemet end 20mA. 

\begin{enumerate}
	\item Maks belastning af transformer er 20VA
	\item Nominel spænding på primær siden er 24V
	\item Nominel spænding på sekundær siden afhænger af trin, transformerne har 8 trin fra 0 til 8V.
	\item Skal minimum levere 500mA til systemet. 
\end{enumerate}

\subsubsection{Måleenhed}
Til en måleenhed burde der laves nøjagtigheds beregninger, og defineres præcision. Dette er der bestemte standarder til, men da det ikke er noget vi har haft undervisning i og vores projekt er "proof of concept". Er det besluttet at måleenheden skal kunne måle indenfor en bestemt procent af det rigtige værdig.

\begin{enumerate}
	\item Måle spændingen ved trinskifteren og forbrugerne mellem 0 og 8 Vrms
	\item Måle spændingen med en præcision på $\pm$ 5\%
	\item Måle strømmen ved trinskifteren og forbrugerne mellem 0 og 500mA
	\item Måle strømmen med en præcision på $\pm$ 5\%
	\item Måle og beregne power factor med en præcision på $\pm$ 5$\%$
	\item Beregne THD med en præcision på $\pm$ 5$\%$
\end{enumerate} 

\subsubsection{kommunikation}

