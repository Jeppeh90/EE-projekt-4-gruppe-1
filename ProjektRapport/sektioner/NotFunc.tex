% !TEX root = ../prj4projektrapport.tex
% SKAL STÅ I TOPPEN AF ALLE FILER FOR AT MASTER-filen KOMPILERES 

\section{Ikke funktionelle krav}
I dette afsnit beskrives de ikke funktionelle krav. Der findes ikke funktionelle krav til trintransformeren, målingerne og kommunikationen. Disse beskrives hver især i følgende afsnit.

\subsubsection{Trintransformer}
Den valgte trintransformer har nedenstående specifikationer. Det er besluttet ikke at lave større strømforbrug i systemet end 500mA. 

\begin{enumerate}
	\item Maks belastning af transformer er 20VA
	\item Nominel spænding på primær siden er 24V
	\item Nominel spænding på sekundær siden afhænger af trin, transformerne har 8 trin fra 1 til 8V.
	\item Skal minimum kunne levere 500mA til systemet. 
\end{enumerate}

\subsubsection{Måleenhed}
Til en måleenhed burde der laves nøjagtighedsberegninger og defineres præcision. Dette er der bestemte standarder til, men da gruppen ikke har modtaget undervisning i emnet, og projektet er proof of concept, er det besluttet, at Måleenheden skal kunne måle indenfor en bestemt procent af den rigtige værdi og ikke gå længere ned i nøjagtighedsbegreberne.

\begin{enumerate}
	\item Måle spændingen ved Trinskifteren og forbrugerne mellem 0 og 8 Vrms
	\item Måle spændingen med en præcision på $\pm$ 5\%
	\item Måle strømmen ved Trinskifteren og forbrugerne mellem 0 og 500mA
	\item Måle strømmen med en præcision på $\pm$ 5\%
	\item Måle og beregne power factor med en præcision på $\pm$ 5$\%$
	\item Beregne THD med en præcision på $\pm$ 5$\%$
\end{enumerate} 

\subsubsection{Kommunikation}
For at sikre at der udvikles en pålidelig kommunikation, stilles der krav til, at der i de fleste tilfælde ikke må ske fejl, men da det er proof of concept, er det valgt ikke at have fejlhåndtering, hvis der skulle ske fejl i en forsendelse.

\begin{enumerate}
	\item Forsinkelsen på brugergrænsefladen ift. ændringer i målte værdier må ikke overstige 2,5 sekunder. 
	\item 95\% af alle sendte data skal være korrekte og uden forstyrrelser. 
\end{enumerate} 

