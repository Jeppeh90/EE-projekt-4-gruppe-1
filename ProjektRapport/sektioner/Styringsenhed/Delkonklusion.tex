% !TEX root = ../../prj4projektrapport.tex
% SKAL STÅ I TOPPEN AF ALLE FILER FOR AT MASTER-filen KOMPILERES 

\section{Delkonklusion}

Styringsenheden har haft tre gennemgående fokuspunkter; sikkerhed og overskuelighed, pålidelig kommunikation og mulighed for nem udvidelse af softwaren til data fra flere Måleenheder.


I kontrolmodulet var Trinskifter en funktionsblok, der på sikker vis bragte systemet rundt mellem sekvenser, herfor opdelt i tilstande, trin og trinskift. Samtidig skulle kommunikationen være pålidelig, hvilket førte til valget af TCP og separate netværk, der styrer data sendt, og hvor data skal gemmes. De separate netværk gør det også nemt at udvide med flere kommandoer og steder at gemme modtaget data.


Brugergrænsefladen blev udviklet med henblik på brugervenlighed og simplicitet samtidig med, at den indeholder mange informationer om systemet. Det er her også vist, at der er mulighed for udvidelse. Dog er der en fysisk begrænsning i forhold til plads på skærmen brugt i produktet.


Kommunikationsmodulet har løst opgaven om at lave en overgang fra en kommunikationsform, UART, til en anden, TCP. Dette platformsinterface er et kritisk punkt for den pålidelige kommunikation. Her er koden igen udviklet, så flere enheder kan tilkobles. En ændring i opsætning af UART forbindelse vil gøre, at stadig kun en Arduino er nødvendig for at muliggøre tilføjelsen af mange flere Måleenheder.
