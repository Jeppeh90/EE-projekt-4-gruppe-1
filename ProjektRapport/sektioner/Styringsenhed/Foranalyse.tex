% !TEX root = ../../prj4projektrapport.tex
% SKAL STÅ I TOPPEN AF ALLE FILER FOR AT MASTER-filen KOMPILERES 

\section{Foranalyse for Styringsenhed}

\subsection{Kontrolmodul og Brugergrænseflade}
Det er en oplagt mulighed at anvende en PLC og et HMI, som Kontrolmodul og Brugergrænseflade i projektet. Dette skyldes at der, som krav til projektet skulle anvendes relevante faglige elementer fra semestrets kurser. Faget Instrumentering og Automatisering(IOA) omhandler netop brugen af PLC og HMI i industrien. Hermed kommer to gode argumenter for PLC og HMI; relevant faglige viden anvendes og i "den virkelige verden" ville det være oplagt at bruge lignende enheder til et projekt som dette.

PLC'en kan programmeres i flere forskellig sprog. Ladder er dog blevet valgt, fordi det er det mest anvendte i IOA. Desuden er det et krav til projektet at en bruger skal kunne interagere med systemet, hvilket opnåes gennem HMI'et.

Kravet om pålidelig transmission af data mellem udvalgte enheder understøttede også en PLC, da denne indeholder mulighed for LAN kommunikation gennem en RJ-45 port.

\subsection{Kommunikationsmodul}
Krav: Systemet skal omfatte pålidelig transmission af data mellem udvalgte enheder.
Komunikationsmodul + shields
Kommunikationsformer
%Arduino udviklingsmiljø

