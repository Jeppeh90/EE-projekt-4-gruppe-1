	\documentclass[a4paper, 11pt,oneside,openany, danish]{memoir} % Starter dokumentet af klassen memoir


%%%%%%%%%%%%%%%%%%%%%%%
		       % PREAMBLE %			
%%%%%%%%%%%%%%%%%%%%%%%

% Papirstørrelse og margener
\usepackage[paper=a4paper, hmargin=1.1in, vmargin=1.1in]{geometry}

% Font encoding og sprog
\usepackage[T1]{fontenc}					% Output encoding
\usepackage[utf8]{inputenc}				% Input encoding
\usepackage[danish]{babel}				% Sprog (orddeling)
\renewcommand{\danishhyphenmins}{22} 	% bedre orddeling, minimum to tegn før og efter deling
\usepackage{lmodern}  					% gør underscores pænere
\usepackage{microtype} 					% laver micro ændringer i text for at udgå luft og orddeling

%% Forside text
%\usepackage{soul} % lege lege
%\sodef\an{}{0.2em}{.9em plus.6em}{1em plus.1em minus.1em}
%\newcommand\stext[1]{\an{\scshape#1}}

% Fyldetekst (Lorem ipsum)
\usepackage{blindtext}

% Biblatex til referencer
\usepackage[backend=bibtex]{biblatex}
\addbibresource{bibfil.bib}

% Tabeller
\usepackage{booktabs}
\usepackage{threeparttable}
\usepackage[tableposition=top]{caption}
\usepackage{tabularx}

%matematik
\usepackage{amsmath,amssymb,mathtools,bm}
\newcommand{\tsub}[1]{_{\textup{#1}}}
\def\doubleunderline#1{\underline{\underline{#1}}}
\usepackage[separate-uncertainty = true,multi-part-units=single]{siunitx}

% XColor: Farver
\usepackage[svgnames,dvipsnames,x11names]{xcolor}

% Figurer og floats
\usepackage[]{graphicx}
\graphicspath{{figurer/}}
\usepackage{placeins}
\usepackage{float}			% Muliggoer eksakt placering af floats, f.eks. \begin{figure}[H]

%%% Tegning af kasser
%\usepackage{calc,graphicx,color}
%\definecolor{mygreen}{rgb}{0,0.6,0}
%\definecolor{mygray}{rgb}{0.5,0.5,0.5}

% Biblatex til referencer
%\usepackage[backend=biber]{biblatex}
%\addbibresource{bib.bib}

% Hyper ref
\usepackage[ unicode=true, colorlinks=false, linktocpage=true, 
pdfborder={0 0 0}, pdfstartpage=1, pdfstartview=FitV, breaklinks=true,
pdfpagemode=UseNone, pageanchor=true, pdfpagemode=UseOutlines,
plainpages=false, bookmarksnumbered, bookmarksopen=true,
bookmarksopenlevel=1, hypertexnames=true, pdfhighlight=/O, urlcolor=Black,
linkcolor=Black, citecolor=Black]{hyperref}

% Clever ref
\usepackage{cleveref}



\settocdepth{subsection}
\setsecnumdepth{subsection}

% Sidetal
\let\footruleskip\undefined
\usepackage{fancyhdr}
\usepackage{lastpage}
\pagestyle{fancy} 
\fancyhf{} 

\fancyhead[R]{\leftmark}
\fancyfoot[R]{\thepage \hspace{0.008in} af \pageref{LastPage}}

\fancypagestyle{}{
	\renewcommand{\headrulewidth}{0pt}
	\fancyhf{}
	\fancyfoot[R]{\thepage \hspace{0.008in} af \pageref{LastPage}}%
	
}

% Starten på dokumentet
\begin{document}


%%%%%%%%%%%%%%%%%%%%%%%
		       % FORSIDEN %			
%%%%%%%%%%%%%%%%%%%%%%%

% !TEX root = ../prj4projektdokumentation.tex
% SKAL STÅ I TOPPEN AF ALLE FILER FOR AT MASTER-filen KOMPILERES 
\thispagestyle{empty}
{\centering
	{\scshape\LARGE Aarhus Universitet \par}
	\vspace{1cm}
	{\scshape\Large 4. semesterprojekt gruppe 1\par}
	{\scshape\Large Projektdokumentation\par}
	\vspace{1.5cm}
	{\huge\bfseries Spændingsregulator\par}
	\vspace{2cm}
	{\Large
		201509249 - Caroline Møller Sørensen\\
		201611140 - Sophia Amailie Mortensen\\
		201505195 - Dennis Slot Larsen \\
		201505115 - Laurids Givskov Jørgensen\\
		201508333 - Søren Jensen\\
		13114 - Jeppe Hansen\\  }
	\vfill
	Vejleder\par
	Emir Pasic
	
	\vfill
	
	{\large \today\par}
	\par}





\frontmatter

%%%%%%%%%%%%%%%%%%%%%%%
             % RESUME & ABSTRACT %			
% !TEX root = ../prj4projektrapport.tex
% SKAL STÅ I TOPPEN AF ALLE FILER FOR AT MASTER-filen KOMPILERES 

Denne rapport beskriver et fjerde semester projekt ved ASE på studieretningen Elektrisk Energiteknologi. Problemstillingen der arbejdes med er, at udvikle et system som kan sikre et stabilt spændingsniveau på en distributionslinje med varierende belastning.

Rapporten beskriver en prototype på en spændingsregulator, som kan installeres på en eksisterende distributionslinje. Prototypen indeholder en simulering af en distributionslinje, hvorpå spændingsregulatorens funktionalitet demonstreres. Reguleringen af spændingen fortages med en trintransformer, som skifter trin på baggrund af data fra måleenheder placeret centralt og decentralt på distributionslinjen.

Projektet indeholder programmering af måleenheder på PSOC, styring/regulering af trintransformeren på en Siemens PLC, og TCP-kommunikation. 
% !TEX root = ../prj4projektrapport.tex
% SKAL STÅ I TOPPEN AF ALLE FILER FOR AT MASTER-filen KOMPILERES 

This paper describes a $4^{th}$ semester project at Aarhus School of Engineering in the field of Electical Engineering. The thesis is devoted to the development of a system that ensures a stable voltage level on a distribution line with varying load.

The paper describes a prototype of a voltage regulator, which can be installed on an existing distribution line. The prototype contains a simulation of a distribution line, on which the functionality of the voltage regulator is demonstrated. The voltage regulation is performed with a step transformer that changes step based on data from measurement devices placed centrally and decentrally on the distribution line. 

The project includes programming of measurement devices on PSoC, controlling of the step transformer on a Siemens PLC and TCP-communication.


%%%%%%%%%%%%%%%%%%%%%%%


%%%%%%%%%%%%%%%%%%%%%%%
         % INDHOLDSFORTEGNELSE %			
\include{sektioner/Indholdsfortegnelse}
%%%%%%%%%%%%%%%%%%%%%%%

\tableofcontents

%%%%%%%%%%%%%%%%%%%%%%%
                        % KAPITLER %			
% !TEX root =../prj4projektrapport.tex

\section{Forord}

\textbf{Praktisk information:} I dette projekt deltog seks ingeniørstuderende fra Ingeniørhøjskolen Aarhus. De studerende er på 4. semester på studieretningen Elektrisk Energiteknologi. Projektgruppens vejleder er Emir Pasic, der løbende har vejledt gruppen. Semesterprojektets afleveringsdato er 29/5-2017 og bedømmelsesdato er 28/6-2017. 

\textbf{Læsevejledning:} Henvisninger til projektdokumentationen er lavet med fodnoter, der angiver kapitelnummer og navn på det afsnit, der henvises til. 

\textbf{Tak til:} Der skal tilskrives en tak til Michael Rangård, Specialist Planlægning, og Poul Bagger Thomsen, Afdelingsleder Anlæg 20/0,4kV, fra Eniig.

% !TEX root = ../prj4projektdokumentation.tex
% SKAL STÅ I TOPPEN AF ALLE FILER FOR AT MASTER-filen KOMPILERES 


\chapter{Termliste}

\begin{table}[htbp]
	\centering
	\begin{tabular}{|l|l|}
		\hline
		\textbf{Term} 	& \textbf{Beskrivelse} \\\hline
		Trinskifter	& Transformer med variabelt omsætningsforhold \\\hline
		Centralt	& Ved trinskifter \\\hline
		Decentralt 	& Ved forbrugeren \\\hline
		
	\end{tabular}
	\caption{Termbeskrivelse}
	\label{tab:termbeskrivelsen}
	
\end{table}
% !TEX root = ../../prj4projektrapport.tex
% SKAL STÅ I TOPPEN AF ALLE FILER FOR AT MASTER-filen KOMPILERES 

I arkitekturen bliver produktet, der kan løse problemformuleringen udformet. I kravsspecifikationen er det blevet gjort klart, hvilke krav der er til systemet. Ud fra kravene er et blok definitionsdiagram og to interne blokdiagrammer blevet lavet, som viser blokkene og deres interne forbindelser i den overordnede løsning.

Et allokeringsdiagram er med til at underbygge forståelsen for, hvordan disse blokke bliver realiseret softwaremæssigt, og hvilke enheder der vil være omdrejningspunkt for hver blok.

I rapporten er kun den endelige og vigtigste arkitektur medtaget. Der henvises til dokumentationen for den resterende del af arkitekturen i projektet.\footnote{Projektdokumentation, 5, Arkitektur}
% !TEX root = ../prj4projektrapport.tex
% SKAL STÅ I TOPPEN AF ALLE FILER FOR AT MASTER-filen KOMPILERES 

Dette kapitel indeholder en overordnet beskrivelse af systemet, samt en gennemgang af de funktionelle og ikke funktionelle krav der stilles til systemet. 

\section{Systembeskrivelse}
Spændingsregulatoren skal være i stand til at analysere forholdene på distributionslinjen. Derfor udvikles en måleenhed, som kan placeres decentralt, ved hver forbruger, og centralt ved spændingsregulatoren. Måleenhederne skal sende værdier for spænding, strøm, powerfactor og indhold af harmoniske til et system, som regulerer spændingen på baggrund af disse målinger. 

Spændingsreguleringen laves med en trintransformer, hvor der med en styringsenhed kan skiftes mellem trinene. Styringsenheden kan automatisk regulere spændingen jf. data fra måleenhederne, eller den kan styres manuelt på en grafisk brugergrænseflade. 

Systemet der er udviklet i dette projekt er proof of concept, så spændingsniveauet er skaleret ned.  Der er designet et simuleringskredsløb i form af en distributionslinje og en række forbrugere, for at vise funktionaliteten af Spændingsregulatoren. Forbrugerene  kan kobles til/fra nettet, for at generere et spændingsfald der giver anledning til en regulering af trintransformeren. 


% !TEX root = ../../prj4projektrapport.tex
% SKAL STÅ I TOPPEN AF ALLE FILER FOR AT MASTER-filen KOMPILERES 

\section{Foranalyse for Styringsenhed}

\subsection{Kontrolmodul og Brugergrænseflade}

IOA kode
PLC og HMI

\subsection{Kommunikationsmodul}

Komunikationsmodul + shields
Kommunikationsformer
%Arduino udviklingsmiljø


\include{sektioner/Arkitektur}
\include{sektioner/DesignOgImplementering}
\include{sektioner/ResultatOgDiskussion}
\include{sektioner/FremtidigtArbejde}
% !TEX root = ../prj4projektrapport.tex
% SKAL STÅ I TOPPEN AF ALLE FILER FOR AT MASTER-filen KOMPILERES 


Projektets formål har været at undersøge, hvordan spændingen kan holdes stabil i et system med varierende belastninger. Dette blev løst ved at udvikle en Spændingsregulator, der kan skifte spændingsniveauer på en trintransformer og derved sikre en stabil spænding hos forbrugerne, selvom forbruget ændres. Regulering på lavspændingsnettet har dog vist sig ikke at være relevant i det danske distributionssystem, men projektets overordnede proof of concept er gennemført. Det er implementeret, at Spændingsregulatoren kan observere THD i systemet. Dette har dog også vist sig ikke at være nødvendigt på distributionsnettet. Det kan derfor konkluderes, at det på nuværende tidspunkt ikke er relevant at implementere Spændingsregulatoren i det danske elnet, men at det i fremtiden måske kan blive en del af en smart grid løsning. Projektgruppen har dog løst problemformuleringen og opfyldt kravene til projektet. Gruppen er desuden stolte af, at have fået systemet til at fungere automatisk, selvom det var nedprioriteret i MoSCoW'en. Gennem projektforløbet har gruppen fået kendskab til problematikker og udfordringer i det danske elnet samt fået erfaringer inden for kommunikationsprotokoller, PLC programmering og påvirkning fra harmoniske.  

Gruppen har desuden opnået en indsigt, der har gjort, at alle medlemmer har kunne se mulige forbedringer til den udviklede prototype, som ville være oplagte at arbejde videre med, hvis projektet skulle fortsættes.

Der er lavet et grundigt forarbejde og en fornuftigt arbejdsfordeling, hvilket har medført, at tidsplanen er overholdt og et tilfredsstillende resultat er opnået. Opgaver i forbindelse med udviklingen af Spændingsregulatoren har været uddelt i tre hold. Der har været god kommunikation holdene imellem, hvilket har givet en forholdsvis nem integrationsfase uden de store udfordringer. Samlet set har gruppen fået et godt udbytte af projektet både fagligt og inden for projektstyring.



 
\include{sektioner/Udviklingsvaerktoejer}
\include{sektioner/Bibliografi}
%%%%%%%%%%%%%%%%%%%%%%%
\mainmatter



\printbibliography

\end{document}


